\documentclass[a4paper,12pt,french]{article}
\RequirePackage[l2tabu, orthodox]{nag}
\usepackage{setspace}
\usepackage[margin=1in]{geometry}
\usepackage{amsmath}
\usepackage{amssymb}

%Enable fr support
\usepackage[utf8]{inputenc}
\usepackage[T1]{fontenc}
\usepackage{lmodern} % load a font with all the characters
\usepackage{babel}
\usepackage{courier}
\usepackage{listings}

%Assign document variables
\date{\today}
\author{Kevin Belisle \& Simon Bernier St-Pierre}
\title{TP2}
\newcommand{\Teacher}{Marc Feeley}
\newcommand{\ClassNum}{IFT-2035}
\newcommand{\ClassName}{Concepts des langages de programmation}
\newcommand{\DateMMMMYYYY}{Décembre 2015}
\newcommand{\Author}{Kevin Belisle}
\newcommand{\Authorr}{Simon Bernier St-Pierre}
\makeatletter

%Custom Header & Footer
\usepackage{fancyhdr}
\pagestyle{fancy}
\fancyhf{}
\fancyhead[L]{\@title}
\fancyhead[R]{\thepage}
\fancyfoot[L]{Kevin Belisle \& Simon Bernier St-Pierre}
\fancyfoot[R]{\DateMMMMYYYY}
\renewcommand{\footrulewidth}{0.4pt}% default is 0pt

% code listings
\lstset{language=C}
\lstset{basicstyle=\footnotesize\ttfamily}

\begin{document}
\begin{titlepage}
	\begin{center}
		\textsc{\normalsize Université de Montréal}\\[2.5cm]

		\textsc{\LARGE \@title}\\[2.5cm]

		\textsc{\small Par}\\[0.25cm]
		\textsc{\LARGE \Author}\\[0.25cm]
		\textsc{\normalsize (20018469)}\\[0.25cm]
		\textsc{\LARGE \Authorr}\\[0.25cm]
		\textsc{\normalsize (20031840)}\\[2.5cm]

		\textsc{\normalsize Baccalauréat en informatique}\\
		\textsc{\normalsize Faculté des arts et des sciences}\\[2.5cm]

		\textsc{\small Travail présenté à \Teacher}\\
		\textsc{\small Dans le cadre du cours \ClassNum}\\
		\textsc{\small \ClassName}\\[2.5cm]

		\textsc{\normalsize \DateMMMMYYYY}\\[1.5cm]
	\end{center}
\end{titlepage}
\section{Fonctionnement général du programme}
\section{Résolution des problèmes}
	\renewcommand{\thesubsection}{(\alph{subsection})}
	\subsection{Comment se fait l'analyse syntaxique d'une expression et comment est réalisé le traitement d'une expression de longueur quelconque}
        La liste de caractères est séparée en liste de mots à l'aide de la fonction \lstinline$tokens$. Cette fonction sépare une liste de caractères sur les espaces. L'analyseur d'expressions est séparé en plusieurs petites fonctions répondant à une interface identique. Les fonctions prennent en paramètre le mot courant, le reste de la liste de mots et une pile contenant les sous-expressions. La fonction \lstinline$dispatch$ choisit, à partir du premier élément de la liste de mots, quelle fonction d'analyse appeler. Le résultat est un arbre de noeuds représentant l'expression donnée. Les noeuds sont représentés par une paire. Le \lstinline$car$ est un symbole représentant le type du noeud, et le \lstinline$cdr$ représente les arguments (données additionelles).\\

        L'évaluation d'expressions se fait à l'aide de plusieurs petites fonctions en forme itérative ainsi que de continuations. Les fonctions d'évaluation répondent aussi à une interface commune, elle prennent les arguments d'un noeud, la liste des variables, et une continuation qui doit être appelée avec les données produites, soit un nombre et une liste de variables. L'évalutation débute par le noeud \og racine \fg{}. Ses sous noeuds sont évalués en envoyant des continuations qui modifieront le résultat des sous expressions et appèleront la continuation du noeud parent. Lorsque qu'on atteint les feuilles de l'arbre, on a terminé de parcourir l'arbre. Les feuilles appèlent la continuation qui leur a été donnée, et on remonte donc dans l'arbre à l'envers, chaque continuation modifiant le réultat et la liste des variables au besoin.
        \subsection{Comment se fait le calcul de l'expression}
	\subsection{Comment se fait l'affectation aux variables}
	\subsection{Comment se fait l'affichage des résultats et erreurs}
        Lorsqu'une erreur est détectée (question 2e), la fonction d'affichage d'erreur est appelée. Cette fonction contient un dictionnaire d'erreurs (symbole) vers message (string). Le symbole est cherché parmis le dictionnaire et la chaîne de caractères trouvée est convertit en liste de caractères et est retournée.
	\subsection{Comment se fait le traitement des erreurs}
        Les erreurs sont représentées par un symbole qui débute par \lstinline$err-$, par exemple \lstinline$err-invalid-varname$ qui signifie qu'un nom de variable invalide a été trouvé. Lorsqu'une erreur est détectée, la fonction retourne un symbole approprié pour l'erreur. Si une fonction retourne un symbole, on sait qu'une erreur est survenue, lors de l'exécution et on affiche un message approprié (question 2d).
\section{Expérience de développement C vs Scheme}
\end{document}
