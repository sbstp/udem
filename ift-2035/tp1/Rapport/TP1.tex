\documentclass[a4paper,12pt,french]{article}
\RequirePackage[l2tabu, orthodox]{nag}
\usepackage{setspace}
\usepackage[margin=1in]{geometry}
\usepackage{amsmath}
\usepackage{amssymb}

%Enable fr support
\usepackage[utf8]{inputenc}
\usepackage[T1]{fontenc}
\usepackage{lmodern} % load a font with all the characters
\usepackage{babel}
\usepackage{courier}
\usepackage{listings}

%Assign document variables
\date{\today}
\author{Kevin Belisle \& Simon Bernier St-Pierre}
\title{TP1}
\newcommand{\Teacher}{Marc Feeley}
\newcommand{\ClassNum}{IFT2035}
\newcommand{\ClassName}{Concepts des langages de programmation}
\newcommand{\DateMMMMYYYY}{Octobre 2015}
\newcommand{\Author}{Kevin Belisle}
\newcommand{\Authorr}{Simon Bernier St-Pierre}
\makeatletter

%Custom Header & Footer
\usepackage{fancyhdr}
\pagestyle{fancy}
\fancyhf{}
\fancyhead[L]{\@title}
\fancyhead[R]{\thepage}
\fancyfoot[L]{Kevin Belisle \& Simon Bernier St-Pierre}
\fancyfoot[R]{\DateMMMMYYYY}
\renewcommand{\footrulewidth}{0.4pt}% default is 0pt

% code listings
\lstset{language=C}
\lstset{basicstyle=\footnotesize\ttfamily}

\begin{document}
\begin{titlepage}
	\begin{center}
		\textsc{\normalsize Université de Montréal}\\[2.5cm]

		\textsc{\LARGE \@title}\\[2.5cm]

		\textsc{\small Par}\\[0.25cm]
		\textsc{\LARGE \Author}\\[0.25cm]
		\textsc{\normalsize (20018469)}\\[0.25cm]
		\textsc{\LARGE \Authorr}\\[0.25cm]
		\textsc{\normalsize (20031840)}\\[2.5cm]

		\textsc{\normalsize Baccalauréat en informatique}\\
		\textsc{\normalsize Faculté des arts et des sciences}\\[2.5cm]

		\textsc{\small Travail présenté à \Teacher}\\
		\textsc{\small Dans le cadre du cours \ClassNum}\\
		\textsc{\small \ClassName}\\[2.5cm]

		\textsc{\normalsize \DateMMMMYYYY}\\[1.5cm]
	\end{center}
\end{titlepage}
\section{Fonctionnement général du programme}

	\subsection{Instructions de compilation}
	Le programme a été testé sur Windows et Linux. Le programme devrait compiler
	sans erreurs et sans avertissements sur Linux. La version du standard C
	utilisée est celle que gcc utilise par défaut soit -std=c89 (gcc 4.8).\\

	Il est possible d'activer les tests unitaires avec \lstinline$#define TEST$.

\section{Résolution des problèmes}
	\renewcommand{\thesubsection}{(\alph{subsection})}

	\subsection{Comment les nombres et variables sont représentés}
		Les nombre est représentés selon une liste chaînée simple.
\begin{lstlisting}
	struct num {
	    bool isNeg;
	    struct digit *first;
	    int refcount;
	};
	struct digit {
	    int val;
	    struct digit *next;
	};
\end{lstlisting}
		La structure \textit{num} représente la tête de la liste chaînée. elle contient le signe stocké dans un booléen, un boléen est notre structure personalisée représentant les valeurs \og false \fg{}(0) or \og true \fg{}(1) sur 1 bit, le premier \textit{digit} du nombre et un compteur de référence.(Plus de détails sur le compteur de référence dans la section 2. c). Les chiffres du nombre sont stockés dans un \textit{digit}. Le \textit{digit} est composée d'un eniter positif de taille 4 bit pour la valeur du chiffre et d'un pointeur vers le prochain chiffre. Les \textit{digits} sont stockés en ordre croissant de puissance.\\

		Les variables sont réprésentés par un tableau de pointeur de \textit{num} stocké dans la structure \textit{inter} qui est fourni à l'ASA pour le calcul de la réponse. Avec les méthodes \lstinline$inter_get_var$ et \lstinline$inter_set_var$, il est possible d'obtenir ou de stocker un \textit{num} à l'index de la lettre de la variable (où a=0,b=1,...,z=25)  en utilisant le compteur de référence. (Plus de détails sur le compteur de référence dans la section 2. c).

	\subsection{Comment se fait l’analyse de chaque ligne et le calcul de la réponse}
		La fonction \lstinline$read_line$ lit l'expression entrée par l'utilisateur un caractère à la fois,
		avec la fonction getchar. Si le caractère est EOF, on arrête la lecture et on retourne
		une erreur signalant EOF. Le programme fera son nettoyage final et terminera. Tant que
		le caractère n'est pas un saut de ligne, les caractères sont copiés dans un tableau
		alloué avec malloc. Lorsque plus d'espace est nécessaire, le tableau est redimensionné.
		Lorsque la fin de ligne et atteinte, on retourne la chaîne de caractères.\\

		La chaîne est ensuite envoyée à la fonction \lstinline$ast_parse$ qui utilise un tokenizer pour séparer
		les éléments du texte en fonction des espaces. Elle itère à travers les tokens et produit un
		arbre de syntaxe abstraite. Elle utilise une pile pour empiler les opérandes, et en fonction
		de l'arité du prochain élément, désempilera le nombre d'opérandes nécessaire. Un nouveau
		noeud sera créé et empilé en fonction du type d'élément. Si le nombre d'opérandes requises
		n'est pas disponible, une erreur est détectée. La procédure détecte les expressions malformées
		et signale ces erreurs. Lorsque tous les éléments on étés analysés, on regarde la hauteur de
		la pile. Si plus qu'un élément si trouve, cela veut dire que plusieurs expressions étaient
		contenues dans le texte reçu. On signale l'erreur. Si ce n'est pas le cas, on désempile le dernier
		élément de la pile, on signale qu'il n'y a pas d'erreur et on obtient l'arbre complet.\\

		La procédure appelant vérifie si \lstinline$ast_parse$ a retourné une erreur. Si c'est le cas, on affiche
		un message approprié. Si ce n'est pas le cas, on envoie l'ASA à la fonction \lstinline$inter_eval$ qui évaluera
		l'expression. L'interprète parcourt l'arbre avec un algorithme récursif, effectue les opérations,
		maintient la liste des variables, et retourne la valeur de l'expression. L'interprète peut aussi
		retourner une erreur si l'allocation échoue ou si une variable indéfinie est utilisée. Si une erreur
		survient, elle sera affichée. Si aucune erreur ne survient, la valeur retournée est affichée.
		Finalement, la mémoire utilisé par l'ASA est libérée et on recommence la saisie.\\

	\subsection{Comment se fait la gestion de la mémoire}
		Les nombres utilisent un compteur de référence qui est maintenu manuellement. Grossièrement,
		c'est l'équivalent d'un \lstinline$shared_ptr$ C++. Les nombres sont créés lors de l'analyse de la
		ligne. Lorsque les nombres sont assignés à des variables, ont augmente le compteur de
		référence. Lorsque l'ASA est libéré, on décrémente le compteur de référence des nombres.
		Les nombres n'ayant pas été assignés à des variables seront donc libérés.\\

		Les structures contenant des données allouées dynamiquement sont aussi allouées de cette
		façon. Cela permet d'avoir d'une initialization complète et exhaustive des structures.
		Si une erreur d'allocation survient, les allocations précédentes sont libérés, et il
		n'existe donc pas d'objet dont l'état est incertain.\\

		L'entête <setjmp.h> n'étant pas permis, la gestion des erreurs d'allocation se fait de
		façon manuelle. Les erreurs d'allocation sont signalées par un pointeur NULL lorsque
		possible. Quand ça n'est pas possible, on utilise plustôt un code d'erreur approprié.
		(Il y a plus de détails sur les erreurs dans la section 2. e) .\\
	\subsection{Comment les algorithmes d’addition, soustraction et multiplication sont implantés}
		Les algorithmes d'addition, de soustraction et de multiplication utilise les algorithmes de calcul \og naïfs \fg{}.
		Cependant, l'algorithme d'addition et de soustraction redirige le calcul si les nombres sont de signes opposés. Voici les quatre cas possibles où a est le terme de gauche et b le terme de droite.
		\[ (1) : a + b \equiv a - -b\]
		\[ (2) : a + -b \equiv a - b  \]
		\[ (3) : -a + b \equiv -(a-b) \]
		\[ (4) : -a + -b \equiv -(a+b) \]
		\\Par exemple, si l'algorithme doit traiter -12 + 15, alors l'algorithme redirige à l'algorithme de sosutraction le calcul 15 -12 à la place. Aussi, l'algoritme de soustraction inversera l'opération si $|a|$ < $|b|$ tq $-(b-a)$.\\

		Les algorithmes d'addition et de soustraction calcule les \textit{digit} un à un avec la notion de surplus/retenue et complète le résultat avec les \textit{digits} restants, si nécesssaire. L'algortihme de soustraction, si nécessaire, supprime tous les zéros de puissance supérieur au \textit{digit} de plus grande puissance non-zéros (0000546 devient 546).\\

		L'algorithme de multiplication itère sur les sommes de puissance de 10 de b (567 devient 500+60+7). Par la suite, il multiplie chaque puissance de 10 de b par a pour obtenir un résultat partiel. Finalement, il additionne toutes les sommes partielles pour obtenir le résultat final.
	\subsection{Comment se fait le traitement des erreurs}
		Les erreurs sont représentées avec des types de données algébriques (aussi appelés  \og tagged union \fg{}
		ou \og variante \fg{}), inspirés par la gestion d'erreur du langage Rust. Un enum est utilisé comme
		discriminant et une structure composée du discriminant et d'une union est utilisée pour représenter
		le résultat d'une opération pouvant échouer. L'union contient des données additionelles sur le
		résultat.\\

		Par exemple, le résultat produit par \lstinline$inter_eval$ est \lstinline$inter_eval_result$. Le discriminant
		est \lstinline$inter_eval_err$ et contient trois options: aucune erreur, erreur d'allocation, variable
		non définie. L'union du résultat contient 2 options: le nombre lorsqu'il n'y a pas d'erreur et
		le nom de la variable lorsqu'une variable non définie est utilisée.\\

		Les erreurs sont passées par copie car leur allocation ne doit pas échoucher, et que leur taille
		est petite de toute façon.\\

		Lorsque la boucle d'exécution reçoit un résultat qui est une erreur, le résultat est envoyé à la
		fonction d'affichage appropriée. Cette dernière affichera l'erreur ainsi que les données additionelles
		contenu par l'union. Si le résultat n'est pas une erreur, la boucle extrait les données dont elle
		a besoin et continue sont travail.
\end{document}
