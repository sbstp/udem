\documentclass[a4paper,12pt]{article}
\RequirePackage[l2tabu, orthodox]{nag}
\usepackage{setspace}
\usepackage[margin=1in]{geometry}
\usepackage{amsmath}
\usepackage{amssymb}

%Enable fr support
\usepackage[utf8]{inputenc}
\usepackage[T1]{fontenc}
\usepackage{lmodern} % load a font with all the characters

%Assign document variables
\date{\today}
\author{Kevin Belisle \& Simon Bernier St-Pierre}
\title{TP1}
\newcommand{\Teacher}{Marc Feeley}
\newcommand{\ClassNum}{IFT2035}
\newcommand{\ClassName}{Concepts des langages de programmation}
\newcommand{\DateMMMMYYYY}{Octobre 2015}
\newcommand{\Author}{Kevin Belisle}
\newcommand{\Authorr}{Simon Bernier St-Pierre}
\makeatletter

%Custom Header & Footer
\usepackage{fancyhdr}
\pagestyle{fancy}
\fancyhf{}
\fancyhead[L]{\@title}
\fancyhead[R]{\thepage}
\fancyfoot[L]{Kevin Belisle \& Simon Bernier St-Pierre}
\fancyfoot[R]{\DateMMMMYYYY}
\renewcommand{\footrulewidth}{0.4pt}% default is 0pt


\begin{document}
\begin{titlepage}
	\begin{center}
		\textsc{\normalsize Université de Montréal}\\[2.5cm]

		\textsc{\LARGE \@title}\\[2.5cm]

		\textsc{\small Par}\\[0.25cm]
		\textsc{\LARGE \Author}\\[0.25cm]
		\textsc{\normalsize (20018469)}\\[0.25cm]
		\textsc{\LARGE \Authorr}\\[0.25cm]
		\textsc{\normalsize (Ton Matricule)}\\[2.5cm]

		\textsc{\normalsize Baccalauréat en informatique}\\
		\textsc{\normalsize Faculté des arts et des sciences}\\[2.5cm]

		\textsc{\small Travail présenté à \Teacher}\\
		\textsc{\small Dans le cadre du cours \ClassNum}\\
		\textsc{\small \ClassName}\\[2.5cm]

		\textsc{\normalsize \DateMMMMYYYY}\\[1.5cm]
	\end{center}
\end{titlepage}
\section{CECI EST UNE SECTION }
	\[Sum: S = A \bigoplus B\]
	\[Carry: C_{out} = AB\]
	\begin{tabular}{cc | cc}
		A&B&$S$&$C_{out}$\\
		\hline
		0&0&0&0\\
		0&1&1&0\\
		1&0&1&0\\
		1&1&0&1\\
	\end{tabular}
\section{Résolution des problèmes}
	\subsection{(b) comment se fait l’analyse de chaque ligne et le calcul de la réponse}
	La fonction read\_line lit l'expression entrée par l'utilisateur un caractère à la fois,
	avec la fonction getchar. Si le caractère est EOF, on arrête la lecture et on retourne
	une erreur signalant EOF. Le programme fera son nettoyage final et terminera. Tant que
	le caractère n'est pas un saut de ligne, les caractères sont copiés dans un tableau
	alloué avec malloc. Lorsque plus d'espace est nécessaire, le tableau est redimensionné.
	Lorsque la fin de ligne et atteinte, on retourne la chaîne de caractères.

	La chaîne est ensuite envoyée à la fonction ast\_parse qui utilise un tokenizer pour séparer
	les éléments du texte en fonction des espaces. Elle itère à travers les tokens et produit un
	arbre de syntaxe abstraite. Elle utilise une pile pour empiler les opérandes, et en fonction
	de l'arité du prochain élément, désempilera le nombre d'opérandes nécessaire. Un nouveau
	noeud sera créé et empilé en fonction du type d'élément. Si le nombre d'opérandes requises
	n'est pas disponible, une erreur est détectée. La procédure détecte les expressions malformées
	et signale ces erreurs. Lorsque tous les éléments on étés analysés, on regarde la hauteur de
	la pile. Si plus qu'un élément si trouve, cela veut dire que plusieurs expressions étaient
	contenues dans le texte reçu. On signale l'erreur. Si ce n'est pas le cas, on désempile le dernier
	élément de la pile, on signale qu'il n'y a pas d'erreur et on obtient l'arbre complet.

	La procédure appelant vérifie si ast\_parse a retourné une erreur. Si c'est le cas, on affiche
	un message approprié. Si ce n'est pas le cas, on envoie l'ASA à la fonction inter\_eval qui évaluera
	l'expression. L'interprète parcourt l'arbre avec un algorithme récursif, effectue les opérations,
	maintient la liste des variables, et retourne la valeur de l'expression. L'interprète peut aussi
	retourner une erreur si l'allocation échoue ou si une variable indéfinie est utilisée. Si une erreur
	survient, elle sera affichée. Si aucune erreur ne survient, la valeur retournée est affichée.
	Finalement, la mémoire utilisé par l'ASA est libérée et on recommence la saisie.

\section{Preuve par induction}
Étape de Base:\\
	\[1^2 = \frac{(-1)^{1+1}1(1+1)}{2} = \frac{1\bullet 1 (2)}{2} = 1\]
Étape Inductive:\\
	Supposons que le formule est vrai pour $n$.\\
	Prouvons que c'est aussi vrai pour $n+1$.
	\[1^2 - 2^2 + 3^2 - ... + (-1)^{n+1}n^2 + (-1)^{n+2}(n+1)^2 = \frac{(-1)^{n+1}n(n+1)}{2}  + (-1)^{n+1+1}(n+1)^2 \]
	\[= \frac{(-1)^{n+1}n(n+1)}{2}  + \frac{2(-1)^{n+1+1}(n+1)^2}{2}\]
	\[= \frac{(-1)^{n+1}n(n+1)}{2}  + \frac{2\bullet-1(-1)^{n+1}(n+1)^2}{2}\]
	\[= \frac{(-1)^{n+1}n(n+1)  -2(-1)^{n+1}(n+1)^2}{2}\]
	\[= \frac{(-1)^{n+1}(n+1)(n  -2(n+1))}{2}\]
	\[= \frac{(-1)^{n+1}(n+1)(-n-2)}{2}\]
	\[= \frac{(-1)^{n+1}(n+1) -1 (n+2)}{2}\]
	\[= \frac{(-1)^{n+2}(n+1)(n+2)}{2}\]
	Alors, la formule est vrai $n+1$ et donc, $1^2 - 2^2 + 3^2 - ... + (-1)^{n+1}n^2 = \frac{(-1)^{n+1}n(n+1)}{2}$
\section{Théorème 1}
	Soit
	\[A = \{a, b, c\}\]
	\[X = P(A)\]
	\[x R y \leftrightarrow x \subseteq y \]
	\[ x < y \leftrightarrow x \subset y  \]
	Prouver que $ x < y \leftrightarrow x \ngeq y$\\
	\linebreak
	\[ x < y \Longrightarrow x \subset y\]
	\[ \Longrightarrow \exists a \in y , a \notin x \]
	\[ \Longrightarrow y \not \subseteq x \]
	\[ \Longrightarrow y \not\leq x \]
\newpage

\end{document}
